\chapter*{Introduction}
\addcontentsline{toc}{chapter}{Introduction}
\pagestyle{plain}


Il Modello Standard (MS) delle interazioni fondamentali � una delle teoria pi� dibattute degli ultimi anni ed ha ottenuto nell'ultimo secolo numerose conferme sperimentali, con altissimi livelli di precisione. L'ultimo pezzo mancante per la conferma della teoria � stato per un lungo periodo il bosone di Higgs, il quanto del campo scalare ritenuto responsabile della rottura spontanea della simmetria di \emph{gauge} del MS. Grazie a questo meccanismo tutte le particelle elementari acquistano massa.\\ La massa del bosone di Higgs � un parametro libero della teoria e pu� variare in un ampio intervallo di valori. Numerosi esperimenti  hanno cercato segni della sua esistenza, ma sono riusciti solo ad escludere alcuni intervalli di massa. Il \emph{Large Hadron Collider} (LHC) � l'acceleratore di particelle costruito per investigare la rottura spontanea della simmetria e riuscire a dare una prova definitiva dell'esistenza del bosone di Higgs.
